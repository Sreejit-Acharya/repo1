\documentclass[11pt]{article}

\usepackage{amsmath,amsfonts}
\usepackage{tabularray}
\usepackage{graphicx}
\usepackage{subfigure}
\usepackage{multirow}

\pagestyle{plain}

\title{Introduction to LaTeX}
\author{Sreejit Acharya \& Srijan Acharya}

\begin{document}
\maketitle
\date{\today}


\section{Introduction}
LaTeX is a high-quality typesetting system; it includes features designed for the production of technical and scientific documentation. LaTeX is the de facto stander for the communication and publication of scientific documents.
\subsection{Background}
LaTeX uses plain text as input and produces formated documents as output. It is widely used in academia for the communication of scientific ideas in the form of articles, books , theses and presentations.

\section{Equations}
a. $E = mc^{2}$ \newline
b. $x = (-b \pm \sqrt{(b^{2}-4ac)})/(2a)$ \newline
c. $\int_{0}^{\pi}\sin(x) = 2$ \newline
d. $\sum_{n=1}^{\propto} 1/n^{2} = \pi^{2}/6$ \newline
e. $$\lim_{x \rightarrow 0}\frac{\sin x}{x} = 1$$ \newline
f.
\left(
\begin{tabular}{cc}
 1&2\\ 3&4\\ 
 \end{tabular}
 \right)
 
 g. $ \vec{A} \cdot \vec{B} = |A||B|\cos\theta$ 
 
 \begin{thebibliography}
 
 \end{thebibliography}
\end{document}